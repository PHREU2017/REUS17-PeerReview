\section{Abstract}
\begin{comment}


We created a Bayesian Network that models and tracks human behavior.  The data set that we used consists of health records from 5,115 black and white, males and females, ages 18-30, called CARDIA data.\cite{Friedman1988}  Using the CARDIA data we were able to track behavioral changes amongst patients, through the implementation of our Bayesian Network.  Our Network then, modeled these changes and made statistical inferences based off different attributes within our network.  These inferences can aid physicians in building a patient specific decision support model. Hospitals can then use these decision support models to produce a more personalized medical approach for treating their patients.
\end{comment}


Coronary Heart Disease (CHD) has had a serious impacted  amongst the American population.  Given the concern for Heart health, a predictive measure has been producted called Coronary Artery Calcification (CAC).  We modeled CAC in a few simple Bayesian Networks using data gathered in the Coronary Artery Risk Development in Young Adults (CARDIA) study.  Using our Bayesian Networks, we implemented three different scoring functions, taking from it, the most accurate model. We took into account both clinical and non-clinical data to determine how shifts in life-style decisions made by young adults can manipulate CAC levels later in life. \cite{Yang}.  Uncovering changes in CAC levels enables us to accurately predict the possible risk of cardiovascular health issues later in life.  This makes it possible to construct a more individualized treatment approach in young adults to alleviate the risk later.  





