\section{Abstract}
\begin{comment}


We created a Bayesian Network that models and tracks human behavior.  The data set that we used consists of health records from 5,115 black and white, males and females, ages 18-30, called CARDIA data.\cite{Friedman1988}  Using the CARDIA data we were able to track behavioral changes amongst patients, through the implementation of our Bayesian Network.  Our Network then, modeled these changes and made statistical inferences based off different attributes within our network.  These inferences can aid physicians in building a patient specific decision support model. Hospitals can then use these decision support models to produce a more personalized medical approach for treating their patients.
\end{comment}


Coronary Heart Disease (CHD) is a leading cause of death in the United States. There are many risk factors of CHD, and these risk factors have complex interactions over the course of decades. Coronary Artery Calcification (CAC) is one indicator of CHD. We used score-based, constraint-based, and local discovery algorithms to learn Bayesian Network (BN) structures for each year of observed data (0, 5, 7, 10, 15, 20) in the Coronary Artery Risk Development in Young Adults (CARDIA) study. These networks model the influence of various clinical and non-clinical risk factors on CAC levels. After comparing the BNs, we selected the most accurate model of the data. Models such as the one selected could enable physicians to construct a more individualized treatment approach for young adults, reducing their risk of CHD later in life.  





