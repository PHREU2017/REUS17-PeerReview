\section{Methods}

\subsection{Data}

The data used was gathered through the Coronary Artery Risk Development in Young Adults (CARDIA) study. This study followed 5115 subjects from 1985-6 until present. Birmingham, AL; Chicago, IL; Minneapolis, MN; and Oakland, CA served as centers for data collection. Each location recruited participants in a way that ensured an even distribution of sex, race, education level, and age-group (18-25 or 25-30). Data gathered from participants included physical measurements, clinical tests, and an in depth questionnaire about lifestyle and socioeconomic status.  The specifics of the study procedures are detailed elsewhere \cite{Friedman1988}.  The breadth of recorded features and longitudinal nature of this study make is a good data set for studying the development of heart disease.  Having such an in depth data set enables us to pin point risk factors in young adults that have the capability to cause serious cardiovascular complications later on.  Providing physicians with the knowledge to install treatments that entail preventative actions in order to improve their patients quality of life in their later years.

The data we analyze comes from years 0 (1985-6), 5 (1990-1), 7 (1992-3), 10 (1995-6), 15 (2000-1), and 20 (2005-6).  We explored risk features such as: sex, age, race, education, cholesterol, bmi, cdl level, ldl level, triglycerides, diastolic bp, systolic bp, glucose, exercise, blood pressure, and smoking. (the features can be viewed in Table~\ref{tab:table1}.)  We split sex, race, and CAC levels into boolean values.   Smoker status was placed into 3 categories: non-smoker, smoker, and heavy smoker.  Also, there is 3 categories for blood pressure:  low, medium, and high.  The data from the rest of the features are continuous.  We discretized the data to make it easier to produce accurate results while implementing it into our BNs.  To do this we cut the data into Quintiles, which means the data was split on the following percentiles based on frequency: 0,0.2, 0.4, 0.8, and 1.0.

In the data cleaning process, we found that it would be most beneficial to exclude some of the subjects from our analyze.  There were two reasons of doing this.1) If the participant's CAC levels were unobserved in year 20. 2) If one or more of a participant's features were not recorded for any year of the study. Otherwise, missing data was replaced by the mean of that patient's data from every observed year.  


\begin{table}{}
\centering
\begin{tabular}{ |c|c|c| } 
 \hline
 Feature & Divisions\\ 
 \hline
 sex & 2\\
 \hline
 race & 2\\
 \hline
 cac & 2\\
 \hline
 hbp & 3\\
 \hline
 smoker & 3\\
 \hline
 age & Quintiles\\
 \hline
 education & Quintiles\\
 \hline
 heavy exercise & Quintiles\\
 \hline
 moderate exercise & Quintiles\\
 \hline
 bmi & Quintiles\\
 \hline
 triglicerides & Quintiles\\
 \hline
 cholesterol & Quintiles\\
 \hline
 ldl & Quintiles\\
 \hline
 hdl & Quintiles\\
 \hline
 glucose & Quintiles\\
 \hline
 dbp & Quintiles\\ 
 \hline
 sbp & Quintiles\\ 
 \hline
 
 
 \hline
\end{tabular}
\caption{Features analyzed used in creating the Bayesian Networks along with the data divisions.}
\label{tab:table1}

\end{table}
 % this is from Early Predictions. Subject to change
 
\subsection{Creating Bayesian Networks}

For Bayesian Network structure learning from the data, we used the bnlearn package in R \cite{Scutari2009, scutari2017package}. The only ordering we used was defining that the sex, race, and age nodes have no parents. 

We used four algorithms for learning the structure of the Bayesian Network. Hill-climbing was the score-based algorithm used. For both of these, we used the optimized implementation to decrease the number of repeated tests within the learning process \cite{Daly}. The score metrics used in the hill-climbing algorithm were AIC \cite{Akaike1974}, BIC \cite{Schwarz1978}, and BDe \cite{Heckerman1995}. 
 
 
We also implemented three constraint based algorithms for learning the structure of the Bayesian Network.  The three constraint based algorithms we used were Incremental Association, Chow Liu, and Semi-Interleaved HITON-PC. Incremental Association and Semi-interleaved Hiton PC algorithms learn the equivalence class of a directed acyclic graph (DAG) from the data set that is presented.\cite{scutari2017package}  They use conditional independence tests to determine the Markov blankets of the attributes, which are utilized to calculate the structure of the Bayesian network. \cite{scutari2017package}  Where as, Chow Liu discovers simple tree structures from a data set using pairwise mutual information coefficients.\cite{scutari2017package}  

For every year data was collected (0, 5, 7, 10, 15, 20), we created a Bayesian Network modeling the features in Table~\ref{tab:table1} using each of the above algorithms. We printed out all of these networks and compared the dependency connections. 


\begin{comment}
\begin{table}{}
\centering
\begin{tabular}{ |c|c|c| } 
 \hline
 Feature Name & Type & Threshold\\ 
 \hline
 Marital Status & Boolean &    \\
 \hline
 Employment Status & Boolean &    \\
 \hline
 Home Ownership & Boolean &    \\
 \hline
 Health Insurance Status & Boolean &    \\
 \hline
 Obesity & Boolean &          \\
 \hline
 Difficulty Paying for Basic Necessities & Boolean &   \\
 \hline
 Sex & Boolean &     \\
 \hline
 Electrocardiogram and Echocardiography & Boolean & \\
 \hline
 Coronary calcium & Boolean & \\
 \hline
 Age & Ordinal & 18,24,25,30 \\
 \hline
 Race & Ordinal & 1,8 \\
 \hline
 Education Level & Ordinal & 1,8 \\
 \hline
 Blood Pressure & Ordinal & cell6 \\ 
 \hline
 Chemistries & Ordinal &   \\ 
 \hline
 Anthropometry & Ordinal & 1,500 \\
 \hline
 Medical history & Ordinal & 1,5 \\
 \hline
 Family history & Ordinal & 1,5\\
 \hline
 Physical Activity & Ordinal & 1,10 \\
 \hline
 Nutrient intake/dietary history & Ordinal & 1,10 \\
 \hline
 Psychosocial parameters & Ordinal & 1,10 \\
 \hline
 Pulmonary function & Ordinal & 1,3 \\
 \hline
 Carotid intimal medial thickness & Ordinal & 1,3 \\
 \hline
 Genetic studies & Ordinal & 1,10  \\
 
 
 
 
 
 
 \hline
\end{tabular}
\end{table}

 
We consider only the longitudinal lifestyle and socioeconomic data in our research. Socioeconomic factors included age, sex, race, education level, marital status, employment status, difficulty paying for basic necessities, home ownership, and health insurance status. 

To model how behaviors change over time, we applied the CARDIA data to BN learning algorithms, which used a hill-climbing approach. To score structures we use BIC, Mutual information, and BDe, \cite{Heckerman} because these metrics have shown promising in previous BN research \cite{Yang}.

\end{comment}{}
% outline below
\begin{comment}

\begin{itemize}
    \item Due to the complexity and unpredictability of machine learning, we are currently unable to define any concrete methods that we will be using for this project.  However, throughout this first week of extensive research, we have discovered a few possible  methodologies that have the capability to be implemented. 

\textbf{Dynamic Bayesian}
       
\item  In short, a dynamic Bayesian network associates variables to each other over adjoining time spaces.  We could possible use this method to relate different factors of the patients that have the potential to determine whether or not they will have a heart attack or not.

\textbf{Hierarchical Bayesian}

 \item this is a statistical model that involves multiple layers that are used to evaluate the parameters of the posterior distribution using the Bayesian method.  The latent models integrate to create the hierarchical model, and the Bayes' theorem is used to fuse them with the observed data, and constitute for all the uncertainty that is existent.  We could possible use this method to produce an ordered model of the attributes that the patients may or may not have.  This will help us evaluate the probability of the patient having a heart attack or not, while providing us with a margin of error in our prediction.
 
\textbf{Weka -> Netica}
 
 \item Originally, we began learning how to use Weka to pre-process, classify,  cluster, and visualize random data sets.  After our last meeting with Dr. Natarajan we were advised that we will now be using Netica do manipulate our data sets. 

\textbf{Programming Workplace and Language}

\item We will be coding our project in the java programming language, using Eclipse as our compiler. 
\end{itemize} 
\end{comment}