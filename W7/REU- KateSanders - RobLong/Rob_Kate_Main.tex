%File: formatting-instruction.tex
\documentclass[letterpaper]{article}
\usepackage{aaai}
\usepackage{times}
\usepackage{helvet}
\usepackage{comment}
\usepackage{courier}
\frenchspacing
\setlength{\pdfpagewidth}{8.5in}
\setlength{\pdfpageheight}{11in}
\setlength{\parskip}

\pdfinfo{
/Title (Insert Your Title Here)
/Author (Put All Your Authors Here, Separated by Commas)}
\setcounter{secnumdepth}{0}  
 \begin{document}
% The file aaai.sty is the style file for AAAI Press 
% proceedings, working notes, and technical reports.
%
\title{Formatting Instructions \\for Authors Using \LaTeX{}}
\author{AAAI Press\\
Association for the Advancement of Artificial Intelligence\\
2275 East Bayshore Road, Suite 160\\
Palo Alto, California 94303\\
}


\maketitle
\begin{abstract}
\begin{quote}
Coronary Heart Disease (CHD) is a leading cause of death in the United States. There are many risk factors of CHD, and these risk factors have complex interactions over the course of decades. Coronary Artery Calcification (CAC) is one indicator of CHD. We used score-based, constraint-based, and local discovery algorithms to learn Bayesian Network (BN) structures for each year of observed data (0, 5, 7, 10, 15, 20) in the Coronary Artery Risk Development in Young Adults (CARDIA) study. These networks model the influence of various clinical and non-clinical risk factors on CAC levels. After comparing the BNs, we selected the most accurate model of the data. Models such as the one selected could enable physicians to construct a more individualized treatment approach for young adults, reducing their risk of CHD later in life.  
\end{quote}
\end{abstract}

\setcounter{secnumdepth}{1}
\section{Introduction}
According to the American Heart Association, an estimated 16.5 million Americans over the age of 20 suffer from Coronary Heart Disease (CHD). Approximately 1 in 7 deaths that occur in the United States are as a result of CHD. The most commonly deadly cardiac event associated with CHD is a Myocardial Infarction (MI), often refered to as a heart attack. Each year, approximately 790,000 MIs occur. That means that, on average, an American has a MI every 40 seconds \cite{Benjamin2017}. /

Coronary Artery Calcification (CAC) is predictive of major cardiac events such as myocardial infarction (MI) and death from CHD. Detrano et. all found that doubling CAC levels caused an approximately 25\% increase in the probability of a major cardiac event occurring, a correlation which held true across all races \cite{Detrano2008}. 




To model causes of Coronary Artery Calcification, we use Bayesian Networks.  Using CARDIA data, we constructed multiple Bayesian Networks for each year, displaying the relationships between the given attributes.  We then implemented three different score based algorithms and three structure learning algorithms on our Bayesian Networks, enabling us to make our predictions more accurate.  The score based algorithms we used are Bayesian information criterion (BIC),Akaike information criterion (AIC), and Bayesian Dirichlet equivalence(BDe)  The structure learning algorithms we used are Semi-Interleaved Hiton PC, Chow-Liu, and incremental association.  After implementing all of the algorithms, we constructed twelve more Bayesian networks; for years 15 and 20 we created a union and intersection separately for both the score based and structure learning algorithms, and a union and intersection of all six algorithm together from year 15 and 20.


\setcounter{secnumdepth}{2}
\section{Background}
\subsection{Introduction to Bayesian Networks}

Bayesian Networks (BNs) are probabilistic models often used in Machine Learning. BNs can be used to model real-world data through a system of nodes and edges. Each node represents a feature, such as cholesterol levels or age. Directed edges connect parent nodes to child nodes, qualitatively representing conditional relationships. Each node can have multiple parents and children. In order to make certain assumptions of independence, BNs must be directed, acyclic graphs (DAGs). This means that the direct edges cannot make cycles within the network.\cite{Russell1995} 

The conditional probability table (CPT) attached to each child node details the quantitative effect of the parents. CPTs are useful for looking up the probability of a feature taking a certain value based on observed features. 

\subsection{Structure Learning}
The structure of a BN is often constructed by a domain expert, leaving just the parameters of the network to be learned from the data. In our research, we are learning both the structure and the parameters of the BN using the data. Structure learning can done using search-and-score techniques, constraint-based methods, or a combination of the two. \cite{Vol2012}  

For our purposes, we will focus on search-and-score structure learning. In search-and-score algorithms, many structures are created from the data and scored. The network with the best score is returned as the optimal model for the data.

In this study, we use a Greedy Local Search algorithm called Hill-Climbing (HC). This algorithm starts with an edgeless network. One at a time, a new edge is created and the resulting network is scored. If the score improves, the new edge is kept. This continues until the score no longer improves with new edge additions. At this point, the network is considered to be at the top of a hill.

There are three primary drawback of HC algorithms. One is the tendency to get stuck at a local maximum rather than continuing on to the global maximum, the truly optimal network. Another drawback is getting lost in a plateau, where no direction causes a significantly better score. Ridges can also hinder progress; the searcher zig-zags over the ridge while only making slight progress towards the top of the hill. These problems can be minimized by using random restarts throughout the search process to better explore the space \cite{Russell1995}.

Structure scoring metrics for HC include Log Likelihood (LL), Bayesian Information Criterion (BIC), Akaike Information Criterion (AIC), and Likelihood-Equivalence Bayesian Dirichlet (BDe). Each of these are described in detail below.

 Log Likelihood, is the simplest scoring metric; the top score goes to the structure that fits the data best. The sum is as follows:
 \begin{equation}
    LL(B | D) = \sum_{i=1}^{n}\sum_{j=1}^{q_i}\sum_{k=1}^{r_i}N_{ijk}\log(\frac{N_{ijk}}{N_{ij}})
\end{equation}
 where \textit{n} is number of features, $q_i$ is the total possible configurations of the parents of a particular feature, and $r_i$ is the number of states for a particular feature. $N_{ijk}$ represents the total number of times that a feature takes it's $k^{th}$ value and the the feature's parents are in their $j^{th}$ configuration within the given data.
 
 
 Using LL alone can lead to an excessive number of parameters. This is known as overfitting. An overfitted model is very specific to the data it is trained on and does not generalize well to other data sets. AIC and BIC improve on LL by adding a penalizing term for network complexity \cite{Vol2012}.

AIC was developed by Hirotugu Akaike \cite{Akaike1974} in the early 1970s. The score can be calculated using the following equation:
\begin{equation}
    AIC(B | D) = LL(B | D) - |B|
\end{equation}
where the DAG along which the factorization is made is represented by \textit{B} and the data is represented by \textit{D}. The penalizing term is \begin{math}|B|\end{math}, which represents the network complexity in terms of the number of parameters in \textit{B}.

BIC was created in 1978 by Gideon Schwarz \cite{Schwarz1978} as an alteration on AIC. It can by calculated as:
\begin{equation}
    BIC(B | D) = LL(B | D) - \frac{1}{2}\log(N)|B|
\end{equation}
  The sample size of \textit{B} is represented by \textit{N}. The penalizing factor for BIC is greater than that of AIC.

Unlike BIC and AIC, BDe uses a Bayesian approach to scoring. This method was deveoped by Heckerman, Geiger, and Chickering in 1995 \cite{Heckerman1995}.

\setcounter{secnumdepth}{3}
\section{Related Works}
\subsection{Previous Models of CARDIA Data}
The development of CAC has been modeled by Dynamic Bayesian Networks (DBNs) using data collected in the Coronary Artery Risk Development in Young Adults (CARDIA) study. This temporal model took into account only non-clinical data to identify how life-style decisions young adults make influence CAC levels later in life \cite{Yang}.

CAC level development has also been predicted from CARDIA's measurements of known risk factors such as age, cholesterol, and BMI. This was done using Statistical Relational Learning (SRL) algorithms, specifically Relational Probability Trees (RPTs) and Relational Functional Gradient Boosting (RFGB). The AUC-ROC for RPT was 0.778 $\pm$ 0.02. For RFGB the AUC-ROC was 0.819 $\pm$ 0.01 \cite{Natarajan2013}. 

\section{Methods}

\subsection{Data}

The data used was gathered through the Coronary Artery Risk Development in Young Adults (CARDIA) study. This study followed 5115 subjects from 1985-6 until present. Birmingham, AL; Chicago, IL; Minneapolis, MN; and Oakland, CA served as centers for data collection. Each location recruited participants in a way that ensured an even distribution of sex, race, education level, and age-group (18-25 or 25-30). Data gathered from participants included physical measurements, clinical tests, and an in depth questionnaire about lifestyle and socioeconomic status.  The specifics of the study procedures are detailed elsewhere \cite{Friedman1988}.  The breadth of recorded features and longitudinal nature of this study make is a good data set for studying the development of heart disease.  Having such an in depth data set enables us to pin point risk factors in young adults that have the capability to cause serious cardiovascular complications later on.  Providing physicians with the knowledge to install treatments that entail preventative actions in order to improve their patients quality of life in their later years.

The data we analyze comes from years 0 (1985-6), 5 (1990-1), 7 (1992-3), 10 (1995-6), 15 (2000-1), and 20 (2005-6).  We explored risk features such as: sex, age, race, education, cholesterol, bmi, cdl level, ldl level, triglycerides, diastolic bp, systolic bp, glucose, exercise, blood pressure, and smoking. (the features can be viewed in Table~\ref{tab:table1}.)  We split sex, race, and CAC levels into boolean values.   Smoker status was placed into 3 categories: non-smoker, smoker, and heavy smoker.  Also, there is 3 categories for blood pressure:  low, medium, and high.  The data from the rest of the features are continuous.  We discretized the data to make it easier to produce accurate results while implementing it into our BNs.  To do this we cut the data into Quintiles, which means the data was split on the following percentiles based on frequency: 0,0.2, 0.4, 0.8, and 1.0.

In the data cleaning process, we found that it would be most beneficial to exclude some of the subjects from our analyze.  There were two reasons of doing this.1) If the participant's CAC levels were unobserved in year 20. 2) If one or more of a participant's features were not recorded for any year of the study. Otherwise, missing data was replaced by the mean of that patient's data from every observed year.  


\begin{table}{}
\centering
\begin{tabular}{ |c|c|c| } 
 \hline
 Feature & Divisions\\ 
 \hline
 sex & 2\\
 \hline
 race & 2\\
 \hline
 cac & 2\\
 \hline
 hbp & 3\\
 \hline
 smoker & 3\\
 \hline
 age & Quintiles\\
 \hline
 education & Quintiles\\
 \hline
 heavy exercise & Quintiles\\
 \hline
 moderate exercise & Quintiles\\
 \hline
 bmi & Quintiles\\
 \hline
 triglicerides & Quintiles\\
 \hline
 cholesterol & Quintiles\\
 \hline
 ldl & Quintiles\\
 \hline
 hdl & Quintiles\\
 \hline
 glucose & Quintiles\\
 \hline
 dbp & Quintiles\\ 
 \hline
 sbp & Quintiles\\ 
 \hline
 
 
 \hline
\end{tabular}
\caption{Features analyzed used in creating the Bayesian Networks along with the data divisions.}
\label{tab:table1}

\end{table}
 % this is from Early Predictions. Subject to change
 
\subsection{Creating Bayesian Networks}

For Bayesian Network structure learning from the data, we used the bnlearn package in R \cite{Scutari2009, scutari2017package}. The only ordering we used was defining that the sex, race, and age nodes have no parents. 

We used four algorithms for learning the structure of the Bayesian Network. Hill-climbing was the score-based algorithm used. For both of these, we used the optimized implementation to decrease the number of repeated tests within the learning process \cite{Daly}. The score metrics used in the hill-climbing algorithm were AIC \cite{Akaike1974}, BIC \cite{Schwarz1978}, and BDe \cite{Heckerman1995}. 
 
 
We also implemented three constraint based algorithms for learning the structure of the Bayesian Network.  The three constraint based algorithms we used were Incremental Association, Chow Liu, and Semi-Interleaved HITON-PC. Incremental Association and Semi-interleaved Hiton PC algorithms learn the equivalence class of a directed acyclic graph (DAG) from the data set that is presented.\cite{scutari2017package}  They use conditional independence tests to determine the Markov blankets of the attributes, which are utilized to calculate the structure of the Bayesian network. \cite{scutari2017package}  Where as, Chow Liu discovers simple tree structures from a data set using pairwise mutual information coefficients.\cite{scutari2017package}  

For every year data was collected (0, 5, 7, 10, 15, 20), we created a Bayesian Network modeling the features in Table~\ref{tab:table1} using each of the above algorithms. We printed out all of these networks and compared the dependency connections. 

\section{Findings}
\begin{itemize}
    \item Bayesian Network structures were learned using Hill-Climbing, Grow and Shrink, Incremental Association, and Semi-Interleaved HITON-PC. For the hill-climbing algorithm, three different scoring metrics were used: BDe, BIC, and ML. After training our data we applied each of the scoring functions on our network and found that (insert best method) was the most accurate way to score the data.Thus, we used the results from this network to make our predictions. 
    \item Insert the photos of the best network 
    \item Provide a Graph of the AUC-ROC curve which illustrates a comparison among the different classifiers.
    \item discuss interesting connections in the network
\end{itemize}

\section{Discussions}
We have created a network for predicting CAC levels with an [insert precision/recall]. This network could contribute to early identification of patients at high risk for CHD based on longitudinal EHRs that take into account both clinical and non-clinical information. By highlighting the early causes of CHD, doctors can also better advise young patients on precautionary measures for decreasing the long-term risk of CHD. This could in turn contribute to an overall decrease in deaths attributed to CHD.

Future work will focus on testing this network's predictive abilities on different data sets and expanding the network to encompass more risk factors. 

\section{Results and Discussions}






%\section{Abstract}
\begin{comment}


We created a Bayesian Network that models and tracks human behavior.  The data set that we used consists of health records from 5,115 black and white, males and females, ages 18-30, called CARDIA data.\cite{Friedman1988}  Using the CARDIA data we were able to track behavioral changes amongst patients, through the implementation of our Bayesian Network.  Our Network then, modeled these changes and made statistical inferences based off different attributes within our network.  These inferences can aid physicians in building a patient specific decision support model. Hospitals can then use these decision support models to produce a more personalized medical approach for treating their patients.
\end{comment}


Coronary Heart Disease (CHD) has had a serious impacted  amongst the American population.  Given the concern for Heart health, a predictive measure has been producted called Coronary Artery Calcification (CAC).  We modeled CAC in a few simple Bayesian Networks using data gathered in the Coronary Artery Risk Development in Young Adults (CARDIA) study.  Using our Bayesian Networks, we implemented three different scoring functions, taking from it, the most accurate model. We took into account both clinical and non-clinical data to determine how shifts in life-style decisions made by young adults can manipulate CAC levels later in life. \cite{Yang}.  Uncovering changes in CAC levels enables us to accurately predict the possible risk of cardiovascular health issues later in life.  This makes it possible to construct a more individualized treatment approach in young adults to alleviate the risk later.  






%\section{Introduction}

According to the American Heart Association, an estimated 16.5 million Americans over the age of 20 suffer from Coronary Heart Disease (CHD). Approximately 1 in 7 deaths that occur in the United States are as a result of CHD. The most commonly deadly cardiac event associated with CHD is a Myocardial Infarction (MI), often refered to as a heart attack. Each year, approximately 790,000 MIs occur. That means that, on average, an American has a MI every 40 seconds \cite{Benjamin2017}.

Coronary Artery Calcification (CAC) is predictive of major cardiac events such as myocardial infarction (MI) and death from CHD. Detrano et. all found that doubling CAC levels caused an approximately 25\% increase in the probability of a major cardiac event occurring, a correlation which held true across all races \cite{Detrano2008}. 

The development of CAC has been modeled by Dynamic Bayesian Networks (DBNs) using data collected in the Coronary Artery Risk Development in Young Adults (CARDIA) study. This temporal model took into account only non-clinical data to identify how life-style decisions young adults make influence CAC levels later in life \cite{Yang}.

CAC level development has also been predicted from CARDIA's measurements of known risk factors such as age, cholesterol, and BMI. This was done using Statistical Relational Learning (SRL) algorithms, specifically Relational Probability Trees (RPTs) and Relational Functional Gradient Boosting (RFGB). The AUC-ROC for RPT was 0.778 $\pm$ 0.02. For RFGB the AUC-ROC was 0.819 $\pm$ 0.01 \cite{Natarajan2013}. 



To model causes of Coronary Artery Calcification, we use Bayesian Networks

\begin{comment}


\textbf{ A.) Problem}

	    	a.) Peoples behaviors have a tendency to drastically change over time.
		
	    	b.) These short term changes can play a huge role in the patients health status
		
	    	c.) Short term changes are not easily observable.
		
	    	d.) If a model were to be created displaying these    changes it could help patients make better decisions about their own lives.


\textbf{B.) Obstacles}
	
		a.) creating a complicated longitudinal model is extremely time consuming.
		
		b.) Longitudinal models can sometimes be intractable
		
		c.) There is a lot of background research that needs to be done before creating and implementing such a complex idea
		
\textbf{C.) Dynamic Bayesian Networks}

		a.)  We used a Dynamic Bayesian Network because it enabled us to represent the dependence between variables and to give a concise specification of the 	joint probability distribution. \cite{Russell1995}
		
		b.)  Within the network each node or variable is equipped with a conditional probability table (CPT).  The nodes are then connected by edges that represent direct influence of one node to another.All together, permitting the network to make statistically based 	inferences, in modeling the CARDIA data set. \cite{Russell1995}
		
		c.)  Used Bases rule as our guide
		
			i.)  P(A|B) = P(B|A)P(A) / P(B)
			   


		

\textbf{D.) Technology Hole}
	
		a.)There has been Dynamic Bayesian Networks used to show connections between behavior and disease
		b.) No model has been produced that specifically represents behavior
		
\textbf{E.) Solution}
	
		a.) We have created a preliminary Dynamic Bayesian Network that will aid in tracking and modeling human behavior using CARDIA data.
		
\textbf{F.) Contributions}

		a.) In this paper, we make thing following contributions:
		
		b.) Our research can be used to build a decision support model for hospitals to aid in several tasks including:
		
			i.) resource allocation
			
			ii.) treatment planning
			
			iii.) Prospectively provide physicians with valuable resources required for making informed decisions
			
		c.) Producing a more personalized medical approach for hospitals to use when treating their patients.
		
		d.)The methods we used to model these behavioral changes can serve as a backbone for further modeling related work.



Trying to track the behavior of an individual over time is a daunting and time consuming task. Due to the tendency for behavior to drastically change over time.  These short term behavioral alterations play a huge role in the patients health status, but are extremely hard to observe.This poses a huge problem when trying to make accurate representations of behavioral data. Thus, if a model were to be created displaying these changes it could help patients make better decisions about their own lives.\wfour{sp}{field}

There are complications involved in the process of producing a behavioral model as such.  Just as tracking the behavior of an individual was time consuming, so is creating a longitudinal model.  Longitudinal models are sometimes intractable, making it hard to deal with.  Also, there is copious amounts of background research that needs to be done before constructing and implementing such a complex idea.\wfive{sp}{field}

In order to generate an elaborate model as such we used a Bayesian Network.  The Bayesian Network enabled us to represent the dependence between variables and gave a concise specification of the joint probability distribution.\cite{Russell1995}  Within the network each node or variable is equipped with a conditional probability table (CPT). The nodes are then connected by edges that represent direct influence of one node to another.  This connection permits the network to make statistically based inferences, in modeling the CARDIA data set.\cite{Russell1995}  As a guide for implementing this model we used Bayes rule which is: P(A|B) = P(B|A)P(A) / P(B).

Within our Bayesian network, we used the variable elimination.  Variable elimination is a standard algorithm for computing probability of evidence with regard to a given  Bayesian network \cite{Zhang1996}  This to successfully utilize local structure in the form of determinism \cite{Jensen1990} and context–specific independence\cite{Boutilier1996} to execute inference more systematically. Also, it enables us to answer multiple queries at the same time.   

Currently, there are Bayesian Networks that are used to show connections between behavior and disease, but none that specifically represent behavior.  To fix this problem, we have created a preliminary Bayesian Network that will aid in tracking and modeling human behavior using CARDIA data. 

CARDIA or The Coronary Artery Risk Development in Young Adults Study, looks at the growth and determinants of clinical and sub-clinical cardiovascular disease and its risk components.\cite{Reis2014} This data set records 5,115 black and white, males and females, ages 18-30. \cite{Friedman1988}  CARDIA presents the enrollment and examination methods, the mean levels of blood pressure, total plasma cholesterol, height, weight and body mass index, and the prevalence of cigarette smoking by age, sex, race and educational standing. \cite{Friedman1988}


In this paper, we make the following contributions: our research can be used to build a decision support model for hospitals to aid in several tasks including; resource allocation, treatment planning, and prospectively provide physicians with valuable resources required for making informed decisions.  Resulting in a  more personalized medical approach for hospitals to use when treating their patients.  Also, the methods we used to model these behavioral changes can serve as a backbone for further modeling related work.

\end{comment}

%
\section{Background}
\subsection{Introduction to Bayesian Networks}

Bayesian Networks (BNs) are probabilistic models often used in Machine Learning. BNs can be used to model real-world data through a system of nodes and edges. Each node represents a feature, such as cholesterol levels or age. Directed edges connect parent nodes to child nodes, qualitatively representing conditional relationships. Each node can have multiple parents and children. In order to make certain assumptions of independence, BNs must be directed, acyclic graphs (DAGs). This means that the direct edges cannot make cycles within the network.\cite{Russell1995} 

The conditional probability table (CPT) attached to each child node details the quantitative effect of the parents. CPTs are useful for looking up the probability of a feature taking a certain value based on observed features. 

\subsection{Structure Learning}
The structure of a BN is often constructed by a domain expert, leaving just the parameters of the network to be learned from the data. In our research, we are learning both the structure and the parameters of the BN using the data. Structure learning can done using search-and-score techniques, constraint-based methods, or a combination of the two. \cite{Vol2012}  

For our purposes, we will focus on search-and-score structure learning. In search-and-score algorithms, many structures are created from the data and scored. The network with the best score is returned as the optimal model for the data.

In this study, we use a Greedy Local Search algorithm called Hill-Climbing (HC). This algorithm starts with an edgeless network. One at a time, a new edge is created and the resulting network is scored. If the score improves, the new edge is kept. This continues until the score no longer improves with new edge additions. At this point, the network is considered to be at the top of a hill.

There are three primary drawback of HC algorithms. One is the tendency to get stuck at a local maximum rather than continuing on to the global maximum, the truly optimal network. Another drawback is getting lost in a plateau, where no direction causes a significantly better score. Ridges can also hinder progress; the searcher zig-zags over the ridge while only making slight progress towards the top of the hill. These problems can be minimized by using random restarts throughout the search process to better explore the space \cite{Russell1995}.

Structure scoring metrics for HC include Log Likelihood (LL), Bayesian Information Criterion (BIC), Akaike Information Criterion (AIC), and Likelihood-Equivalence Bayesian Dirichlet (BDe). Each of these are described in detail below.

 Log Likelihood, is the simplest scoring metric; the top score goes to the structure that fits the data best. The sum is as follows:
 \begin{equation}
    LL(B | D) = \sum_{i=1}^{n}\sum_{j=1}^{q_i}\sum_{k=1}^{r_i}N_{ijk}\log(\frac{N_{ijk}}{N_{ij}})
\end{equation}
 where \textit{n} is number of features, $q_i$ is the total possible configurations of the parents of a particular feature, and $r_i$ is the number of states for a particular feature. $N_{ijk}$ represents the total number of times that a feature takes it's $k^{th}$ value and the the feature's parents are in their $j^{th}$ configuration within the given data.
 
 
 Using LL alone can lead to an excessive number of parameters. This is known as overfitting. An overfitted model is very specific to the data it is trained on and does not generalize well to other data sets. AIC and BIC improve on LL by adding a penalizing term for network complexity \cite{Vol2012}.

AIC was developed by Hirotugu Akaike \cite{Akaike1974} in the early 1970s. The score can be calculated using the following equation:
\begin{equation}
    AIC(B | D) = LL(B | D) - |B|
\end{equation}
where the DAG along which the factorization is made is represented by \textit{B} and the data is represented by \textit{D}. The penalizing term is \begin{math}|B|\end{math}, which represents the network complexity in terms of the number of parameters in \textit{B}.

BIC was created in 1978 by Gideon Schwarz \cite{Schwarz1978} as an alteration on AIC. It can by calculated as:
\begin{equation}
    BIC(B | D) = LL(B | D) - \frac{1}{2}\log(N)|B|
\end{equation}
  The sample size of \textit{B} is represented by \textit{N}. The penalizing factor for BIC is greater than that of AIC.

Unlike BIC and AIC, BDe uses a Bayesian approach to scoring. This method was deveoped by Heckerman, Geiger, and Chickering in 1995 \cite{Heckerman1995}.
%\section{Related Works}







\begin{comment}

\subsubsection{Early Prediction of Coronary Artery Calcification Levels Using Statistical Relational Learning}
This study uses two kinds of Statistical Relational Learning (SRL) algorithms for predicting CAC- levels, which are Relational Probability Trees and Relational Functional Gadient Boosting.  These algorithms are applied to the set of CARDIA data, using the measured risk factors to predict the possibility of a patient getting Coronary Heart Disease (CHD) later in life. Being able to predict the likelihood of someone developing CHD, enables  

Previous research with the CARDIA database statistically links behaviors and socioeconomic status with psychosocial vulnerability \cite{ScherwitzL, Scherwitz1992}, physical fitness \cite{Lewis1997, Shishehbor}, and relative risks of heart disease \cite{Karlamangla2005}. 

Bayesian Networks have been used to model the longitudinal links between behavioral and socioeconomic data and Coronary Artery Calcification \cite{Yang}. Machine learning techniques have also been used to predict whether an individual has a rare disease based solely on behavioral data \cite{MacLeod2016}. However, this paper contains the first implementation of a Bayesian Network for modeling solely longitudinal behavioral data.


\subsection{Bayesian Networks - Informative}

\subsubsection{Bayesian Reasoning and Machine Learning}
This textbook gives a very thourough introduction to Bayes Nets and other Machine Learing concepts, written for undergraduate level students. \cite{Barber}
\begin{itemize}
    \item Forward:
    \begin{itemize}
        \item R. R. Bouckaert. Bayesian belief networks: from construction to inference. PhD thesis, University of Utrecht, 1995.
        \item J. Besag and P. Green. Spatial statistics and Bayesian computation. Journal of the Royal Statistical Society, Series B, 55:25–37, 1993.
566
    \end{itemize} 
    \item Backwards:
    \begin{itemize}
        \item  Raghavan, Vasanthan, et al. "Modeling temporal activity patterns in dynamic social networks." IEEE Transactions on Computational Social Systems 1.1 (2014): 89-107.
    \end{itemize}
\end{itemize}

\subsubsection{Where Do the Numbers Come From?}
This article outlines the high-level procedures for creating probabilistic networks, as well as common problems encountered. \cite{Druzdzel}
\begin{itemize}
    \item Forward:
    \begin{itemize}
        \item Norman E. Fenton, Martin Neil, Jose Galan Caballero, "Using Ranked Nodes to Model Qualitative Judgments in Bayesian Networks", Knowledge and Data Engineering IEEE Transactions on, vol. 19, pp. 1420-1432, 2007, ISSN 1041-4347.
        \item Pekka Laitila, Kai Virtanen, "Improving Construction of Conditional Probability Tables for Ranked Nodes in Bayesian Networks", Knowledge and Data Engineering IEEE Transactions on, vol. 28, pp. 1691-1705, 2016, ISSN 1041-4347.
        \item Ye Chen, Divakaran Liginlal, "Bayesian Networks for Knowledge-Based Authentication", Knowledge and Data Engineering IEEE Transactions on, vol. 19, pp. 695-710, 2007, ISSN 1041-4347.
        \item Gao Xiao-guang, Yang Yu, Guo Zhi-gao, Chen Da-qing, "Bayesian approach to learn Bayesian networks using data and constraints", Pattern Recognition (ICPR) 2016 23rd International Conference on, pp. 3667-3672, 2016.
        \item E. Rajabally, P. Sen, S. Whittle, J. Dalton, "Aids to Bayesian belief network construction", Intelligent Systems 2004. Proceedings. 2004 2nd International IEEE Conference, vol. 2, pp. 457-461 Vol.2, 2004.
    \end{itemize} 
    \item Backwards:
    \begin{itemize}
        \item Modeling Coronary Artery Calcification Levels From Behavioral Data in a Clinical Study \cite{Yang}
    \end{itemize}
\end{itemize}

\subsubsection{Learning Bayesian Networks: The Combination of Knowledge and Statistical Data}
This paper outlines the BDe scoring metric for Bayesian Networks. This metric combines event equivalency and parameter modularity to simplify the process of integrating someone's prior knowledge into a BN. \cite{Heckerman}

\begin{itemize}
    \item Forward:
    \begin{itemize}
        \item Cooper, G. & Herskovits, E. (January, 1991). A Bayesian method for the induction of probabilistic networks from data. Technical Report SMI-91-1, Section on Medical Informatics, Stanford University.
        \item Chickering, D. (1995a). A transformational characterization of equivalent Bayesian-network structures. In Proceedings of Eleventh Conference on Uncertainty in Artificial Intelligence, Montreal, QU, pages 87–98. Morgan Kaufmann.
        \item Geiger, D. & Heckerman, D. (1995). A characterization of the Dirichlet distribution with application to learning Bayesian networks. In Proceedings of Eleventh Conference on Uncertainty in Artificial Intelligence, Montreal, QU, pages 196–207. Morgan Kaufmann.
    \end{itemize}
    \item Backwards:
    \begin{itemize}
        \item Lauritzen, Steffen L. Graphical models. Vol. 17. Clarendon Press, 1996.
        \item Witten, Ian H., et al. Data Mining: Practical machine learning tools and techniques. Morgan Kaufmann, 2016.
    \end{itemize}
\end{itemize}

\subsubsection{Aids to Bayesian Belief Network Construction}
An  article outlining various ways to construct BNs. There were links to some potentially useful software applications as well some detail on how to elicit expert opinions for BNs.\cite{Rajabally}

\begin{itemize}
    \item Forward:
    \begin{itemize}
    \item Rajabally, E., Pratyush Sen, S. Whittle, and J. Dalton. 2017. “Aids to Bayesian Belief Network Construction.” In 2004 2nd International IEEE Conference on “Intelligent Systems”. Proceedings (IEEE Cat. No.04EX791), 2:457–61. IEEE. Accessed May 31. doi:10.1109/IS.2004.1344793.
    \item D. Koller, and A Pfeffer, “Objeef-Oriented Bayesian Netwnks” Proceedings of the 13th Annuol Conference on Uncertainty in Artrficiol Intelligence, Rhode Island, 1997, pp.302-313.
    \item M.J. huzdcl, and L.C. van der Gaag “Building probabilistic networks: where do the numbers come from?” IEEE Trmsocbom on Knowledge ondDarn Engineering, vol. 12, no. 4, pp.481486,2000. \cite{Druzdzel}
\end{itemize}
    \item Backwards:
    \begin{itemize}
        \item Wu, C., Yingzi Lin, and Wen-Jun Zhang. "Human attention modeling in a human-machine interface based on the incorporation of contextual features in a Bayesian network." Systems, Man and Cybernetics, 2005 IEEE International Conference on. Vol. 1. IEEE, 2005.
    \end{itemize}
\end{itemize}

\subsubsection{Bayesian Networks without Tears}
This is an introduction to Bayesian Networks with lots of analogies, written for those with limited probability theory background. It might be useful as a reference for explaining concepts in this paper \cite{AmericanAssociationforArtificialIntelligence.1980}

\begin{itemize}
    \item Forward:
    \begin{itemize}
    \item Henrion, M. 1988. Propagating Uncertainty in Bayesian Networks by Logic Sampling. In Uncertainty in Artificial Intelligence 2, eds. J. Lemmer and L. Kanal, 149–163. Amsterdam: North Holland
    \item Cooper, G. F. 1987. Probabilistic Inference Using Belief Networks is NP-Hard, Technical Report, KSL- 87-27, Medical Computer Science Group, Stanford Univ.
Dean,
\end{itemize}
    \item Backwards:
    \begin{itemize}
    \item Baldi, Pierre, and Søren Brunak. Bioinformatics: the machine learning approach. MIT press, 2001.
    \item Buntine, Wray. "A guide to the literature on learning probabilistic networks from data." IEEE Transactions on knowledge and data engineering 8.2 (1996): 195-210.
\end{itemize}
\end{itemize}

\subsubsection{Machine Learning: A Probabilistic Perspective}
A book that goes in depth on the basic math used in creating Bayesian Networks. \cite{Murphy}
\begin{itemize}
    \item Forward:
    \begin{itemize}
    \item Schapire, R. and Y. Freund (2012). Boosting: Foundations and Algo- rithms. MIT Press.
    \item Quinlan, J. R. (1986). Induction of de-
cision trees. Machine Learning 1, 81–106.
\end{itemize}
    \item Backwards:
    \begin{itemize}
    \item Shalev-Shwartz, Shai, and Shai Ben-David. Understanding machine learning: From theory to algorithms. Cambridge university press, 2014.
    \item Jordan, Michael I., and Tom M. Mitchell. "Machine learning: Trends, perspectives, and prospects." Science 349.6245 (2015): 255-260.
\end{itemize}
\end{itemize}

\subsubsection{Bayesian Network Classifiers}
This article explains naive Bayes and Bayesian Networks.  It goes to touch on Bayesian classifiers and how they can be used to eliminate bias. \cite{Friedman1997}
\begin{itemize}
    \item Forward:
    \begin{itemize}
    \item Domingos, Pedro, and Michael Pazzani. "On the optimality of the simple Bayesian classifier under zero-one loss." Machine learning 29.2 (1997): 103-130.
    \item Wu, Xindong, et al. "Top 10 algorithms in data mining." Knowledge and information systems 14.1 (2008): 1-37.
\end{itemize}
    \item Backward:
    \begin{itemize}
    \item Buntine, W. (1996). A guide to the literature on learning probabilistic networks from data. IEEE Trans. on Knowledge and Data Engineering, 8, 195–210.
    \item Bouckaert, R. R. (1994). Properties of Bayesian network learning algorithms. In R. López de Mantarás & D. Poole (Eds.), Proceedings of the Tenth Conference on Uncertainty in Artificial Intelligence (pp. 102–109). San Francisco, CA: Morgan Kaufmann.
\end{itemize}
\end{itemize}



\subsection{Bayesian Networks - Applied}

\subsubsection{Modeling Coronary Artery Calcification Levels From Behavioral Data in a Clinical Study}
Using only socio-demographic & health behavior information from longitudinal EHD, researchers attemped to predict patients' CAC-levels in early to middle adulthood using Dynamic Bayesian Networks. \cite{Yang}


\begin{itemize}
    \item Backwards:
    \begin{itemize}
    \item Koller, D., Friedman, N.: Probabilistic Graphical Models: Principles and Techniques. MIT Press (2009)
4. Liu, Z., Malone, B.M.
    \item Eaton, D., Murphy, K.: Bayesian structure learning using dynamic programming and MCMC. In: UAI (2007)
3.
\end{itemize}
\end{itemize}

\subsection{Modeling Behavior}

\subsubsection{Association of Neighborhood Socioeconomic Status with Physical Fitness in Healthy Young Adults: the CARDIA Study}
The socioeconomic status of the neighborhood someone lives in has an effect on their physical fitness independent of that individual's socioeconomic status. \cite{Shishehbor}

\begin{itemize}
    \item Forward: 
    \begin{itemize}
       \item Popkin BM, Siega-Riz AM, Haines PS. A comparison of dietary trends among racial and socioeconomic groups in the United States. N Engl J Med. 1996; 335(10):716–720. [PubMed: 8703172] 
       \item Hu FB, Manson JE, Stampfer MJ, et al. Diet, lifestyle, and the risk of type 2 diabetes mellitus in women. N Engl J Med. 2001; 345(11):790–797. [PubMed: 11556298] 
       \item . Berkman LF. Tracking social and biological experiences: the social etiology of cardiovascular disease. Circulation. 2005; 111(23):3022–3024. [PubMed: 15956147] 
       \item  Henderson C, Diez Roux AV, Jacobs DR Jr, et al. Neighbourhood characteristics, individual level socioeconomic factors, and depressive symptoms in young adults: the CARDIA study. J Epidemiol Community Health. 2005; 59(4):322–328. [PubMed: 15767387]
        \item  Link BG, Phelan J. Social conditions as fundamental causes of disease. J Health Soc Behav. 1995 Spec No:80–94.
    \end{itemize}

\end{itemize}

\end{comment}
%\section{Methods}

\subsection{Data}

The data used was gathered through the Coronary Artery Risk Development in Young Adults (CARDIA) study. This study followed 5115 subjects from 1985-6 until present. Birmingham, AL; Chicago, IL; Minneapolis, MN; and Oakland, CA served as centers for data collection. Each location recruited participants in a way that ensured an even distribution of sex, race, education level, and age-group (18-25 or 25-30). Data gathered from participants included physical measurements, clinical tests, and an in depth questionnaire about lifestyle and socioeconomic status.  The specifics of the study procedures are detailed elsewhere \cite{Friedman1988}.  The breadth of recorded features and longitudinal nature of this study make is a good data set for studying the development of heart disease.  Having such an in depth data set enables us to pin point risk factors in young adults that have the capability to cause serious cardiovascular complications later on.  Providing physicians with the knowledge to install treatments that entail preventative actions in order to improve their patients quality of life in their later years.

The data we analyze comes from years 0 (1985-6), 5 (1990-1), 7 (1992-3), 10 (1995-6), 15 (2000-1), and 20 (2005-6).  We explored risk features such as: sex, age, race, education, cholesterol, bmi, cdl level, ldl level, triglycerides, diastolic bp, systolic bp, glucose, exercise, blood pressure, and smoking. (the features can be viewed in Table~\ref{tab:table1}.)  We split sex, race, and CAC levels into boolean values.   Smoker status was placed into 3 categories: non-smoker, smoker, and heavy smoker.  Also, there is 3 categories for blood pressure:  low, medium, and high.  The data from the rest of the features are continuous.  We discretized the data to make it easier to produce accurate results while implementing it into our BNs.  To do this we cut the data into Quintiles, which means the data was split on the following percentiles based on frequency: 0,0.2, 0.4, 0.8, and 1.0.

In the data cleaning process, we found that it would be most beneficial to exclude some of the subjects from our analyze.  There were two reasons of doing this.1) If the participant's CAC levels were unobserved in year 20. 2) If one or more of a participant's features were not recorded for any year of the study. Otherwise, missing data was replaced by the mean of that patient's data from every observed year.  


\begin{table}{}
\centering
\begin{tabular}{ |c|c|c| } 
 \hline
 Feature & Divisions\\ 
 \hline
 sex & 2\\
 \hline
 race & 2\\
 \hline
 cac & 2\\
 \hline
 hbp & 3\\
 \hline
 smoker & 3\\
 \hline
 age & Quintiles\\
 \hline
 education & Quintiles\\
 \hline
 heavy exercise & Quintiles\\
 \hline
 moderate exercise & Quintiles\\
 \hline
 bmi & Quintiles\\
 \hline
 triglicerides & Quintiles\\
 \hline
 cholesterol & Quintiles\\
 \hline
 ldl & Quintiles\\
 \hline
 hdl & Quintiles\\
 \hline
 glucose & Quintiles\\
 \hline
 dbp & Quintiles\\ 
 \hline
 sbp & Quintiles\\ 
 \hline
 
 
 \hline
\end{tabular}
\caption{Features analyzed used in creating the Bayesian Networks along with the data divisions.}
\label{tab:table1}

\end{table}
 % this is from Early Predictions. Subject to change
 
\subsection{Creating Bayesian Networks}

For Bayesian Network structure learning from the data, we used the bnlearn package in R \cite{Scutari2009, scutari2017package}. The only ordering we used was defining that the sex, race, and age nodes have no parents. 

We used four algorithms for learning the structure of the Bayesian Network. Hill-climbing was the score-based algorithm used. For both of these, we used the optimized implementation to decrease the number of repeated tests within the learning process \cite{Daly}. The score metrics used in the hill-climbing algorithm were AIC \cite{Akaike1974}, BIC \cite{Schwarz1978}, and BDe \cite{Heckerman1995}. 
 
 
We also implemented three constraint based algorithms for learning the structure of the Bayesian Network.  The three constraint based algorithms we used were Incremental Association, Chow Liu, and Semi-Interleaved HITON-PC. Incremental Association and Semi-interleaved Hiton PC algorithms learn the equivalence class of a directed acyclic graph (DAG) from the data set that is presented.\cite{scutari2017package}  They use conditional independence tests to determine the Markov blankets of the attributes, which are utilized to calculate the structure of the Bayesian network. \cite{scutari2017package}  Where as, Chow Liu discovers simple tree structures from a data set using pairwise mutual information coefficients.\cite{scutari2017package}  

For every year data was collected (0, 5, 7, 10, 15, 20), we created a Bayesian Network modeling the features in Table~\ref{tab:table1} using each of the above algorithms. We printed out all of these networks and compared the dependency connections. 


\begin{comment}
\begin{table}{}
\centering
\begin{tabular}{ |c|c|c| } 
 \hline
 Feature Name & Type & Threshold\\ 
 \hline
 Marital Status & Boolean &    \\
 \hline
 Employment Status & Boolean &    \\
 \hline
 Home Ownership & Boolean &    \\
 \hline
 Health Insurance Status & Boolean &    \\
 \hline
 Obesity & Boolean &          \\
 \hline
 Difficulty Paying for Basic Necessities & Boolean &   \\
 \hline
 Sex & Boolean &     \\
 \hline
 Electrocardiogram and Echocardiography & Boolean & \\
 \hline
 Coronary calcium & Boolean & \\
 \hline
 Age & Ordinal & 18,24,25,30 \\
 \hline
 Race & Ordinal & 1,8 \\
 \hline
 Education Level & Ordinal & 1,8 \\
 \hline
 Blood Pressure & Ordinal & cell6 \\ 
 \hline
 Chemistries & Ordinal &   \\ 
 \hline
 Anthropometry & Ordinal & 1,500 \\
 \hline
 Medical history & Ordinal & 1,5 \\
 \hline
 Family history & Ordinal & 1,5\\
 \hline
 Physical Activity & Ordinal & 1,10 \\
 \hline
 Nutrient intake/dietary history & Ordinal & 1,10 \\
 \hline
 Psychosocial parameters & Ordinal & 1,10 \\
 \hline
 Pulmonary function & Ordinal & 1,3 \\
 \hline
 Carotid intimal medial thickness & Ordinal & 1,3 \\
 \hline
 Genetic studies & Ordinal & 1,10  \\
 
 
 
 
 
 
 \hline
\end{tabular}
\end{table}

 
We consider only the longitudinal lifestyle and socioeconomic data in our research. Socioeconomic factors included age, sex, race, education level, marital status, employment status, difficulty paying for basic necessities, home ownership, and health insurance status. 

To model how behaviors change over time, we applied the CARDIA data to BN learning algorithms, which used a hill-climbing approach. To score structures we use BIC, Mutual information, and BDe, \cite{Heckerman} because these metrics have shown promising in previous BN research \cite{Yang}.

\end{comment}{}
% outline below
\begin{comment}

\begin{itemize}
    \item Due to the complexity and unpredictability of machine learning, we are currently unable to define any concrete methods that we will be using for this project.  However, throughout this first week of extensive research, we have discovered a few possible  methodologies that have the capability to be implemented. 

\textbf{Dynamic Bayesian}
       
\item  In short, a dynamic Bayesian network associates variables to each other over adjoining time spaces.  We could possible use this method to relate different factors of the patients that have the potential to determine whether or not they will have a heart attack or not.

\textbf{Hierarchical Bayesian}

 \item this is a statistical model that involves multiple layers that are used to evaluate the parameters of the posterior distribution using the Bayesian method.  The latent models integrate to create the hierarchical model, and the Bayes' theorem is used to fuse them with the observed data, and constitute for all the uncertainty that is existent.  We could possible use this method to produce an ordered model of the attributes that the patients may or may not have.  This will help us evaluate the probability of the patient having a heart attack or not, while providing us with a margin of error in our prediction.
 
\textbf{Weka -> Netica}
 
 \item Originally, we began learning how to use Weka to pre-process, classify,  cluster, and visualize random data sets.  After our last meeting with Dr. Natarajan we were advised that we will now be using Netica do manipulate our data sets. 

\textbf{Programming Workplace and Language}

\item We will be coding our project in the java programming language, using Eclipse as our compiler. 
\end{itemize} 
\end{comment}
%\section{Findings}
\begin{itemize}
    \item Bayesian Network structures were learned using Hill-Climbing, Grow and Shrink, Incremental Association, and Semi-Interleaved HITON-PC. For the hill-climbing algorithm, three different scoring metrics were used: BDe, BIC, and ML. After training our data we applied each of the scoring functions on our network and found that (insert best method) was the most accurate way to score the data.Thus, we used the results from this network to make our predictions. 
    \item Insert the photos of the best network 
    \item Provide a Graph of the AUC-ROC curve which illustrates a comparison among the different classifiers.
    \item discuss interesting connections in the network
\end{itemize}
%\section{Discussions}
With our findings we can contribute accurate medical predictions that reveal the probability of cardiovascular issues happening later on in a patients life.  Given this knowledge we can aid the medical field in producing treatments that take a more patient-specific approach than before.  Enabling doctors to take precautionary measures earlier in a patients life to prevent cardiovascular problems such as Coronary Heart Disease (CHD) or more specifically Myocardial Infarction (MI) from occurring. Which can ultimately assist in reducing the substantial amount of Americans who suffer from Coronary Heart Diseases. 
%\section{Results and Discussions}

\bibliographystyle{aaai}
\bibliography{sample}

%\balancecolumns 

\end{document}
