\section{Discussion}

We have created a network for predicting CAC levels with an [insert precision/recall]. This network could contribute to early identification of patients at high risk for CHD based on longitudinal EHRs that take into account both clinical and non-clinical information. By highlighting the early causes of CHD, doctors can also better advise young patients on precautionary measures for decreasing the long-term risk of CHD. This could in turn contribute to an overall decrease in deaths attributed to CHD.

Future work will focus on testing this network's predictive abilities on different data sets and expanding the network to encompass more risk factors. 

\begin{comment}
With our findings we can contribute accurate medical predictions that reveal the probability of cardiovascular issues happening later on in a patients life.  Given this knowledge we can aid the medical field in producing treatments that take a more patient-specific approach than before.  Enabling doctors to take precautionary measures earlier in a patients life to prevent cardiovascular problems such as Coronary Heart Disease (CHD) or more specifically Myocardial Infarction (MI) from occurring. Which can ultimately assist in reducing the substantial amount of Americans who suffer from Coronary Heart Disease. 
\end{comment}